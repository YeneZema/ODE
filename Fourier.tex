\documentclass[paper=a4, fontsize=11pt,twoside]{scrartcl}		% KOMA article

\usepackage[a4paper,pdftex]{geometry}										% A4paper margins
\setlength{\oddsidemargin}{5mm}												% Remove 'twosided' indentation
\setlength{\evensidemargin}{5mm}

\usepackage[english]{babel}
\usepackage[protrusion=true,expansion=true]{microtype}	
\usepackage{amsmath,amsfonts,amsthm,amssymb}
\usepackage{graphicx}
\usepackage{microtype}
\usepackage{siunitx}
\usepackage{booktabs}
\usepackage[colorlinks=false, pdfborder={0 0 0}]{hyperref}
\usepackage{cleveref}

\newtheorem{thm}{Theorem}[section]
\newtheorem{lem}[thm]{Lemma}
\newtheorem{prop}[thm]{Proposition}
\newtheorem{cor}[thm]{Corollary}
\newtheorem{conj}[thm]{Conjecture}

\begin{document}
\title{Basel problem using Fourier Series}
\date{7 April, 2015}
\author{Miliyon T.}
\maketitle

\section{Fourier series}
For any function $f(x)$ its fourier representation is given by
\begin{align}
f(x)=\frac{a_0}{2}+\sum_{n=1}^\infty \biggl[ a_{n}\cos\biggl(\frac{n\pi x}{L}\biggl)+b_{n} \sin\biggl(\frac{n\pi x}{L}\biggl) \biggl]\qquad for~-L\leq x\leq L
\end{align}
Where \begin{align*}
a_0&=\frac{1}{L}\int_{-L}^{L} f(x)dx \\
 a_n&=\frac{1}{L}\int_{-L}^{L} f(x)\cos\biggl(\frac{n\pi x}{L}\biggl)dx \\
b_n&=\frac{1}{L}\int_{-L}^{L} f(x)\sin\biggl(\frac{n\pi x}{L}\biggl)dx
\end{align*}
But if $f(x)$ even function, then its fourier representation is as follows
\begin{align}\label{even}
f(x)=\frac{a_0}{2}+\sum_{n=1}^\infty  a_{n}\cos\biggl(\frac{n\pi x}{L}\biggl) \qquad for~-L\leq x\leq L
\end{align}
Where \begin{align*}
a_0&=\frac{1}{L}\int_{-L}^{L} f(x)dx \\
 a_n&=\frac{1}{L}\int_{-L}^{L} f(x)\cos\biggl(\frac{n\pi x}{L}\biggl)dx
\end{align*}



\section{Basel problem}

\begin{thm}\label{thm}
The Fourier series representation of $x^2$  on $ -1\leq x \leq1 $ is
$$ \frac{1}{3}+\sum_{n=1}^{\infty} (-1)^n \frac{4}{n^2\pi^2}\cos(n\pi x)$$
\end{thm}

\begin{proof}
Since $x^2$ is even function from (\ref{even}) we have the following
\begin{align}\label{basel}
x^2=\frac{a_0}{2}+\sum_{n=1}^\infty  a_{n}\cos(n\pi x) \qquad for~-1\leq x\leq 1
\end{align}
Where \begin{align*}
a_0&=\int_{-1}^{1} f(x)dx \\
&=2\int_{0}^{1} x^2 dx \\
&=2\biggl[ \frac{x^3}{3}\biggl]_{0}^{1}\\
&=\frac{2}{3}
\end{align*}

And
\begin{align*}
 a_n&=\int_{-1}^{1} f(x)\cos(n\pi x)dx\\
 &=\int_{-1}^{1} x^2\cos(n\pi x)dx\\
 &=2\int_{0}^{1} x^2\cos(n\pi x)dx
\end{align*}
Now, use integration by parts to evaluate the integral. Finally you will get
\begin{align*}
 a_n&=(-1)^n\frac{4}{n^2\pi^2}
\end{align*}
Substituting $a_0$ and $a_n$ in (\ref{basel}) gives the desired results.
 \begin{align*}
 x^2=\frac{1}{3}+\sum_{n=1}^{\infty} (-1)^n \frac{4}{n^2\pi^2}\cos(n\pi x)
 \end{align*}
\end{proof}

\begin{cor}[Basel Problem]
\begin{align*}
\sum_{n=1}^{\infty}\frac{1}{n^2}=\frac{\pi^2}{6}
\end{align*}
\end{cor}
\begin{proof}
Substitute $x=1$ in Theorem \ref{thm}
\end{proof}


\begin{thebibliography}{6}



{\small

\item[\textbf{[1]}] %Reference updated - 21 August 2010 - TWJ
[Ronald Bracewell] \textit{The Fourier Transform and Its Applications.} Third Edition, 1999.

\item[\textbf{[2]}] %Reference updated - 21 August 2010 - TWJ
[E. M. Stein and G. Weiss]
\textit{Introduction to Fourier Analysis on Euclidean Spaces.}

\item[\textbf{[3]}]
[G. B. Folland]  \textit{Fourier Analysis and Its Applications.} 1992.
}

\end{thebibliography}

\end{document}
