\documentclass[leqno,DIV=calc,paper=a4,fontsize=11pt]{article}

\usepackage[margin=1in]{geometry}
\setlength{\oddsidemargin}{0mm}					% Adjusting margins to center the colorbox, ...
\setlength{\evensidemargin}{0mm}					% ... you might want to change these

\usepackage[english]{babel}
\usepackage[protrusion=true,expansion=true]{microtype}	
\usepackage{amsmath,amsfonts,amsthm,amssymb}
\usepackage{graphicx}
\usepackage{xcolor}
\usepackage{fix-cm}
% various theorems, numbered by section
\usepackage{array}
\usepackage{caption}
\usepackage{multirow}
\usepackage{makecell}%line width

\usepackage{tikz}
\usepackage{verbatim}
\usetikzlibrary{arrows, positioning, quotes, shapes}

\newtheorem{thm}{Theorem}[section]
\newtheorem{lem}[thm]{Lemma}
\newtheorem{prop}[thm]{Proposition}
\newtheorem{cor}[thm]{Corollary}
\newtheorem{conj}[thm]{Conjecture}

\theoremstyle{definition}
\newtheorem{defn}[thm]{Definition}
\newtheorem{defns}[thm]{Definitions}
\newtheorem{con}[thm]{Construction}
\newtheorem{exmp}[thm]{Example}
\newtheorem{notn}[thm]{Notation}
\newtheorem{notns}[thm]{Notations}
\newtheorem{addm}[thm]{Addendum}
\newtheorem{exer}[thm]{Exercise}
\newtheorem{rem}[thm]{Remark}
\theoremstyle{plain}


\theoremstyle{remark}
\newtheorem{rems}[thm]{Remarks}
\newtheorem{warn}[thm]{Warning}
\newtheorem{sch}[thm]{Scholium}
\DeclareMathOperator{\id}{id}
% ------------------------------------------------------------------------------
% Definitions (do not change this)
% ------------------------------------------------------------------------------
\newcommand{\HRule}[1]{\hfill \rule{0.2\linewidth}{#1}} 	% Horizontal rule

\definecolor{grey}{rgb}{0.9,0.9,0.9}

\makeatletter							% Title
\def\printtitle{%						
    {\centering \@title\par}}
\makeatother									

\makeatletter							% Author
\def\printauthor{%					
    {\centering \large \@author}}				
\makeatother							

% ------------------------------------------------------------------------------
% Metadata (Change this)
% ------------------------------------------------------------------------------
\title{	\fontsize{50}{60}\selectfont
			\vspace*{0.7cm}
            \hfill A Note on Differential 	\\[0.8cm]
			\hfill Equations 	\\[0.8cm]
		}

\author{
		\hfill Miliyon T.\\	
		\hfill Addis Ababa University\\	
		\hfill Department of Mathematics\\
        \hfill \texttt{http://www.albohessab.weebly.com} \\
}

\newcommand{\R}[1]{$(\ref{#1})$}
\newcommand{\rrr}{I\!\!R}

\begin{document}
\parindent=0em
% ------------------------------------------------------------------------------
% Maketitle
% ------------------------------------------------------------------------------
\thispagestyle{empty}				% Remove page numbering on this page

\colorbox{grey}{
	\parbox[t]{1.0\linewidth}{
		\printtitle
		\vspace*{0.5cm}
	}
}

  	\vfill
\printauthor								% Print the author data as defined above
\HRule{1pt}
\clearpage
% ------------------------------------------------------------------------------
% Begin document (Content goes below)
% ------------------------------------------------------------------------------
\tableofcontents
\newpage
\section{Introduction}

\begin{defn}[\textbf{Lipschitz}]\label{def1}
A map $f:\Omega\subset \Bbb R\times\Bbb R^n\rightarrow \Bbb R^n$ is Lipschitz in $x$ if the inequality
\begin{align}
|f(t,x_1)-f(t,x_2)|\leq K|x_1-x_2|,
\end{align}
holds wherever $(t,x_1),(t,x_2)\in \Omega.$
\end{defn}
As a consequence of Definition (\ref{def1}) a function f is Lipschitz iff there exists a constant $K>0$ such that
\begin{align*}
\frac{|f(t,x_1)-f(t,x_2)|}{|x_1-x_2|}\leq K,\qquad x_1\neq x_2,
\end{align*}
whenever $(t,x_1),(t,x_2)\in \Omega.$
\begin{thm}
Define rectangle $R$ by
\begin{align*}
R=\{(t,x):|t-t_0|\leq a,\ |x-x_0|\leq b\}
\end{align*}
where $a,b\in \Bbb R.$ If the following two conditions are satisfied
\begin{enumerate}
  \item $f:\Bbb R\rightarrow \Bbb R$ be a real valued continuous function.
  \item $\frac{\partial f}{\partial x}$ be defined and continuous on $R$.
\end{enumerate}
Then $f$ is Lipschitz on $x$.
\end{thm}

\begin{proof}

\end{proof}


\begin{lem}[\textbf{Gronwall}]\label{thm1}
Assume that $f,g:[t_0,\infty)\rightarrow \Bbb R^{+}$ are non negative continuous functions. let $k>0$ be a constant. Then the inequality
\begin{align}
f(t)\leq k+\int_{t_0}^{t}g(s)f(s)ds,\qquad t\geq t_0
\end{align}
implies the inequality
\begin{align}
f(t)\leq k\exp\biggl(\int_{t_0}^{t}g(s)ds\biggl),\qquad t\geq t_0.
\end{align}
\end{lem}
\begin{proof}
By hypothesis we have
\begin{align}\label{gron1}
\frac{f(t)g(t)}{k+\int_{t_0}^{t}g(s)f(s)ds}\leq g(t), \qquad t\geq t_0
\end{align}
Since
\begin{align*}
\frac{d}{dt}\biggl(k+\int_{t_0}^{t}g(s)f(s)ds\biggl)=g(t)f(t)
\end{align*}
By integrating (\ref{gron1}) between the limits $t_0$ and $t$, we have
\begin{align*}
\ln\biggl(k+\int_{t_0}^{t}g(s)f(s)ds\biggl)-\ln(k)\leq\int_{t_0}^{t}g(s)ds
\end{align*}
In other words,
\begin{align*}
k+\int_{t_0}^{t}g(s)f(s)ds\leq k \exp\biggl(\int_{t_0}^{t}g(s)ds\biggl)
\end{align*}
Hence,
\[f(t)\leq k \exp\biggl(\int_{t_0}^{t}g(s)ds\biggl) \]
\end{proof}

\begin{thm}[\textbf{Gronwall}\footnote{T.H. Gronwall proved the inequality in 1919.}]
Let $x,\psi$ and $\chi$ be real continuous functions defined in $I=[a,b]$, $\psi(t)\geq0$ for $t\in I$. Suppose on $I$ the following inequality holds
\begin{align}
x(t)\leq \psi(t)+\int_{a}^{t}\chi(s)x(s)ds
\end{align}
Then
\begin{align}
x(t)\leq \psi(t)+\int_{a}^{t}\chi(s)\psi(s)\exp{\biggl[\int_{s}^{t}\chi(u)du\biggl]}ds
\end{align}
in $I$.
\end{thm}
\begin{proof}

\end{proof}


\begin{cor}
Let $f$ and $k$ be as in Lemma (\ref{thm1}) If the inequality
\begin{align}
f(t)\leq k\int_{t_0}^{t}g(s)ds,\qquad t\geq t_0
\end{align}
holds then,
\begin{align}
f(t)\equiv 0,\text{ for } t\geq t_0.
\end{align}
\end{cor}
\begin{proof}

\end{proof}

\subsection{Picard's Successive Approximations}

\begin{defn}[\textbf{IVP}] \label{ivp}An initial value problem is given by
\begin{align}\label{ivp}
\begin{cases}
x'=f(t,x)\\
x(t_0)=x_0
\end{cases}
\end{align}
\end{defn}

Consider the IVP in (\ref{ivp}), the Picard's approximation to (\ref{ivp}) is given by

\begin{align}
x_{n+1}(t)=x_0+\int_{t_0}^{t}f(s,x_{n}(s))ds
\end{align}

\begin{exmp}
Find the fourth Picard's approximation to the differential equation
\[\frac{dx}{dt}=-x, \qquad x(0)=1, \ t\geq0.\]
\begin{proof}[Solution]

\end{proof}
\end{exmp}

\begin{exmp}
Find the third Picard's approximation to the differential equation
\[\frac{d^2y}{dx^2}=xy+1,\qquad y_0=1, \frac{dy}{dx}=0.\]
\begin{proof}[Solution]
Transform the second order equation $y''=f(x,y,y')$ into first order system,

\begin{align*}
y'&=z\\
z'&=f(x,y,z)\\
y_0&=1\\
z_0&=0
\end{align*}
Then the Picard iteration is given by
\begin{align*}
y_{n+1}&=y_0+\int_{0}^{t}z_{n}(s)ds\\
z_{n+1}&=z_0+\int_{0}^{t}f(s,y_{n}(s),z_{n}(s))ds
\end{align*}
Hence,
\begin{align*}
y_{1}&=1+\int_{0}^{t}0ds\\
     &=1
\end{align*}
\begin{align*}
z_{1}&=0+\int_{0}^{t}f(s,y_{0}(s),z_{0}(s))ds\\
     &=\int_{0}^{t}(s+1)ds
\end{align*}

\end{proof}
\end{exmp}
\begin{exmp}
Find the third Picard's approximation to the differential equation
\[\frac{d^2y}{dx^2}=x^3\biggl(y+\frac{dy}{dx}\biggl),\qquad y_0=1,\ \frac{dy}{dx}=\frac{1}{2}.\]
\begin{proof}[Solution]

\end{proof}
\end{exmp}

\section{Existence and Uniqueness of solution}
\begin{defn}[\textbf{$\varepsilon$-approximate}]
Let $f\in C(\Bbb R)$ on $D$. An $\varepsilon$-approximate solution to the IVP $x'=f(t,x)$ is a function $\Phi\in C$ on a $t$ interval $I$ such that
\begin{enumerate}
  \item $(t,\Phi(t))\in D$,for $t\in I$.
  \item $\Phi(t)\in C^{1}$, except for a finite set $S$ on which $\Phi$ has discontinuity.
  \item $\|\Phi'(t)-f(t,\Phi(t))\|\leq\varepsilon$, for $t\in I\backslash S$.
\end{enumerate}
\end{defn}

\begin{thm}[\textbf{Picard-Lindelof}]
Assume that $f$ is a continuous and Lipschitz w.r.t x on the rectangle
\begin{align*}
R=\{(t,x):|t-t_0|\leq a,\ |x-x_0|\leq b\}.
\end{align*}
Let
\begin{align*}
M:=\max_{(t,x)\in R}\{\|f(t,x)\|\}
\end{align*}
and
\begin{align*}
\alpha:=\min\biggl\{a,\frac{b}{M}\biggl\}
\end{align*}
Then the initial value problem (\ref{ivp}) has a unique solution $x$ on $[t_0,t_0+\alpha]$. Furthermore,
\begin{align*}
|x(t)-x_0|<b,\qquad \text{for } t\in[t_0,t_0+\alpha].
\end{align*}
\end{thm}

\begin{proof}

\end{proof}

\subsection{Cauchy-Peano Theorem}

Let $I=[a,b]\subset \Bbb R$ be an interval and let $F(I,\Bbb R)$ denote the set of all real valued functions defined on $I$.

\begin{defn}[\textbf{Equicontinuous}]
A set of functions $F = \{f\}$ defined on a real interval $I$ is said to be equicontinuous on $I$ if, given any $\varepsilon > 0$, there exists a $\delta_{\varepsilon} > 0$, independent of $f\in F$ and also $t,\widetilde{t}\in I$ such that
\begin{align*}
|f(t)-f(\widetilde{t})|<\epsilon, \text{ whenever } |t-\widetilde{t}| < \delta.
\end{align*}
\end{defn}

\begin{thm}[\textbf{Cauchy-Peano}]
If $f\in C(\Bbb R)$, then there exists a solution $\phi\in C'$ of the IVP $x'=f(t,x)$ on $|t-\tau|<\alpha$ for which $\phi(\tau)=\xi$.
\end{thm}

\begin{proof}

\end{proof}

\begin{thm}[\textbf{Variation of constant formula}]
If $p:(a,b)\to\Bbb R$ and $q:(a,b)\to \Bbb R$ are continuous functions, where $-\infty\leq a<b\leq \infty$. Then the unique solution of the IVP:
\[x'=p(t)x+q(t),\qquad x(t_0)=x_0\]
where $t_0\in (a,b)$, $x_0\in \Bbb R$ is given by
\begin{align}
x(t)=e^{\int_{t_0}^{t}p(r)dr}x_0+e^{\int_{t_0}^{t}p(r)dr}\int_{t_0}^{t}e^{-\int_{t_0}^{t}p(r)dr}q(s)ds,\qquad t\in (a,b).
\end{align}
\end{thm}

\begin{proof}

\end{proof}
\section{Linear systems}
We consider systems of differential equations of the form
\[
x_1'=ax_1+bx_2,
\]

\vspace*{-.35in}
\begin{equation}
\label{twosys}
\end{equation}

\vspace*{-.35in}
\[
x_2'=cx_1+dx_2,
\]
where $a,b,c$, and $d$ are real numbers. At first glance the system
may seem to be first-order; however, it is coupled, and this
two-dimensional system is more closely related to a second-order
differential equation.

Before finding the solution, some notation is introduced to simplify
the form of the system. The new notation implicitly removes most of
the $=$ and $+$ operations. The form of the system suggests arranging
the unknown functions $x_1$, $x_2$ vertically and enclosed in brackets
({\em i.e.} matrix form)
\begin{equation}
\left(
\begin{array}{c}x_1\\x_2\end{array}
\right).
\label{matrix}
\end{equation}
This new object is similar to a number or function. Indeed, operations
may be performed on \R{matrix}. Let $a\in\rrr$. We define
\[
\left(
\begin{array}{c}u\\v\end{array}
\right)' :=
\left(
\begin{array}{c}u'\\v'\end{array}
\right)
\hspace{.5in}
a\left(
\begin{array}{c}u\\v\end{array}
\right)\equiv\left(
\begin{array}{c}u\\v\end{array}
\right)a :=
\left(
\begin{array}{c}au\\av\end{array}
\right).
\]
The new object can be part of mathematical statements. We say
\[
\left(
\begin{array}{c}a\\b\end{array}
\right) =
\left(
\begin{array}{c}c\\d\end{array}
\right)
\]
if and only if $a=c$ and $b=d$. Moreover, we set
\[
\left(
\begin{array}{cc}a&b\\c&d\end{array}
\right)
\left(
\begin{array}{c}x_1\\x_2\end{array}
\right):=
\left(
\begin{array}{c}ax_1+bx_2\\cx_1+dx_2\end{array}
\right).
\]
Then \R{twosys} is equivalent to
\[
\left(
\begin{array}{c}x_1\\x_2\end{array}
\right)'=
\left(
\begin{array}{cc}a&b\\c&d\end{array}
\right)
\left(
\begin{array}{c}x_1\\x_2\end{array}
\right),
\]
often written more simply as
\[
{\bf x}' = \left(
\begin{array}{cc}a&b\\c&d\end{array}
\right){\bf x}.
\]

\begin{center}{\bf\sc Real Eigenvalues}\end{center}

In the calculations and expressions that follow both the matrix form
(above) and the component form will be given and separated by ``or.''
They are equivalent. Based on our discussion in class, we expect a
solution of \R{twosys} to have the form
\begin{equation}
{\bf x} = \left(\begin{array}{c}\xi_1\\ \xi_2\end{array}\right)e^{rt}
\hspace{.25in} \mbox{or} \hspace{.25in}
\begin{array}{cc}x_1=\xi_1e^{rt}\\x_2=\xi_2e^{rt}\end{array}
\label{guess}
\end{equation}
where as of now, $\xi_1$, $\xi_2$, and $r$ are unknown. To be more
concrete, we consider a specific system.

%\pagebreak
\noindent {\bf Example 1}. Consider the system of differential
equations
\begin{equation}
{\bf x}' = \left(
\begin{array}{cc}1&1\\4&1\end{array}
\right){\bf x}
\hspace{.25in} \mbox{or} \hspace{.25in}
\begin{array}{cc}x_1'=x_1+x_2\\x_2'=4x_1+x_2\end{array}.
\label{ex1}
\end{equation}
Our guess, \R{guess}, is substituted in \R{ex1}, and we find
\[
re^{rt}\left(\begin{array}{c}\xi_1\\ \xi_2\end{array}\right)
=
\left(
\begin{array}{cc}1&1\\4&1\end{array}
\right)\left(\begin{array}{c}\xi_1\\ \xi_2\end{array}\right)e^{rt}
\hspace{.25in} \mbox{or} \hspace{.25in}
\begin{array}{cc}r\xi_1=\xi_1+\xi_2\\r\xi_2=4\xi_1+\xi_2\end{array}.
\]
Rearranging gives
\begin{equation}
\left(
\begin{array}{cc}1-r&1\\4&1-r\end{array}
\right)\left(\begin{array}{c}\xi_1\\ \xi_2\end{array}\right)=
\left(\begin{array}{c}0\\0\end{array}\right)
\hspace{.25in} \mbox{or} \hspace{.25in}
\begin{array}{cc}(1-r)\xi_1+\xi_2=0\\4\xi_1+(1-r)\xi_2=0\end{array}.
\label{lines}
\end{equation}

Geometrically the last system of equations describe two lines passing
through the origin. We do not want a unique solution (zero in this
case), for that would say $\xi_1=\xi_2=0$, and our guess would not
produce anything interesting. We want the lines in \R{lines} to be
the same, so we require the slopes to be equal. That is,
\[
r^2-2r-3=0.
\]
The roots, called {\em eigenvalues}, are $r=3,-1$.

Left to find are $\xi_1$ and
$\xi_2$. Inserting $r=3$ back in \R{lines}, we find
\[
\begin{array}{c}-2\xi_1+\xi_2=0\\4\xi_1-2\xi_2=0\end{array}.
\]
As expected this system has infinitely many solutions, one of them is
\[
{\bf \xi} = \left(\begin{array}{c}1\\2\end{array}\right)
\hspace{.25in} \mbox{or} \hspace{.25in}
\begin{array}{c}\xi_1=1\\\xi_2=2\end{array},
\]
and, a solution is
\[
{\bf x} = \left(\begin{array}{c}1\\2\end{array}\right)e^{3t}
\hspace{.25in} \mbox{or} \hspace{.25in}
\begin{array}{c}x_1=e^{3t}\\x_2=2e^{3t}\end{array}.
\]
A similar calculation for $r=-1$ shows
\[
{\bf \xi} = \left(\begin{array}{r}1\\-2\end{array}\right)
\hspace{.25in} \mbox{or} \hspace{.25in}
\begin{array}{c}\xi_1=1\\\xi_2=-2\end{array},
\]
and, a solution is
\[
{\bf x} = \left(\begin{array}{r}1\\-2\end{array}\right)e^{3t}
\hspace{.25in} \mbox{or} \hspace{.25in}
\begin{array}{c}x_1=e^{3t}\\x_2=-2e^{3t}\end{array}.
\]
The general solution is therefore,
\[
{\bf x}=c_1\left(\begin{array}{c}1\\2\end{array}\right)e^{3t}
+c_2\left(\begin{array}{r}1\\-2\end{array}\right)e^{-t}
\hspace{.25in} \mbox{or} \hspace{.25in}
\begin{array}{c}x_1=c_1e^{3t}+c_2e^{-t}\\x_2=c_12e^{3t}-c_22e^{-t}\end{array}.
\]

\pagebreak
\noindent{\bf Example 2}. Find the general solution of the ODE
\[
y''-2y'-3y=0, \hspace{.2in} y(0)=2,\,\,\, y'(0)=2.
\]
using the approach in Example 1.

\vspace{.1in}
\noindent{\em Solution}. We may turn the ODE into a system of ODEs by
setting $x_1=y$ and $x_2=y'$. Then the second-order ODE is equivalent
to
\[
\left(\begin{array}{c}x_1\\ x_2\end{array}\right)'=
\left(
\begin{array}{cc}0&1\\3&2\end{array}\right)
\left(\begin{array}{c}x_1\\x_2\end{array}\right),\hspace{.25in}
{\bf x}(0)=\left(\begin{array}{c}1\\1\end{array}\right).
\]
A comparison of \R{ex1} and \R{lines} in Example 1
shows that we can immediately find the equations for
${\bf \xi}$ in our guess. Here
\[
\left(
\begin{array}{cc}0-r&1\\3&2-r\end{array}
\right)\left(\begin{array}{c}\xi_1\\ \xi_2\end{array}\right)=
\left(\begin{array}{c}0\\0\end{array}\right)
\hspace{.25in} \mbox{or} \hspace{.25in}
\begin{array}{cc}-r\xi_1+\xi_2=0\\3\xi_1+(2-r)\xi_2=0\end{array}.
\]

The polynomial for  $r$ is just the determinate or the upper-left
diagonal element times the lower-right minus the product of the
remaining two diagonal elements. Here, we find $-r(2-r)-3=0$ or
\[
r^2-2r-3=0.
\]
Note that this is the characteristic polynomial for the original
second-order ODE! The roots are $r=3,-1$. Inserting $r=3$ in the
equations for $\xi$, we find
\[
\begin{array}{c}-3\xi_1+\xi_2=0\\3\xi_1-\xi_2=0\end{array}.
\]
As expected this system has infinitely many solutions, one of them is
\[
{\bf \xi} = \left(\begin{array}{c}1\\3\end{array}\right)
\hspace{.25in} \mbox{or} \hspace{.25in}
\begin{array}{c}\xi_1=1\\\xi_2=3\end{array},
\]
The other root, $r=-1$ gives
\[
\begin{array}{c}\xi_1+\xi_2=0\\3\xi_1+3\xi_1=0\end{array},
\]
and
\[
{\bf \xi} = \left(\begin{array}{c}1\\-1\end{array}\right)
\hspace{.25in} \mbox{or} \hspace{.25in}
\begin{array}{c}\xi_1=1\\\xi_2=-1\end{array}.
\]

The general solution is therefore
\[
{\bf x}=c_1\left(\begin{array}{c}1\\3\end{array}\right)e^{3t}
+c_2\left(\begin{array}{r}1\\-1\end{array}\right)e^{-t}
\hspace{.25in} \mbox{or} \hspace{.25in}
\begin{array}{c}x_1=c_1e^{3t}+c_2e^{-t}\\x_2=c_13e^{3t}-c_2e^{-t}\end{array}.
\]
Since we set $y=x_1$ in the original transformation, the top line is
the general solution to the second-order ODE, and the bottom line is the
derivative of the top ($x_2=y'=x_1'$).

To find $c_1$ and $c_2$ we use the initial data -
\[
\left(\begin{array}{c}2\\2\end{array}\right)
= c_1\left(\begin{array}{c}1\\3\end{array}\right)
+c_2\left(\begin{array}{r}1\\-1\end{array}\right).
\]
Solving, we find $c_1=1=c_2$, and the solution is
\[
{\bf x}=\left(\begin{array}{c}1\\3\end{array}\right)e^{3t}
+\left(\begin{array}{r}1\\-1\end{array}\right)e^{-t}
\hspace{.25in} \mbox{or} \hspace{.25in}
\begin{array}{c}y=e^{3t}+e^{-t}\\y'=3e^{3t}-e^{-t}\end{array}.
\]

\pagebreak

\vspace{.5in}
\begin{center}{\bf\sc Complex Eigenvalues}\end{center}
In this section we mostly drop the component form of the equations
(get used to it). As in previous problems we use Euler's formula to
change exponents with complex numbers to oscillatory functions.

If $z=a+ib$ is a complex number, the complex conjugate is
$\bar{z}=a-ib$. That is, $i$ is replaced with $-i$. In addition,
the real and imaginary parts are denoted
\[
\mbox{Re}(z) = a \hspace{.5in} \mbox{Im}(z) = b.
\]
Both are real numbers.

\vspace{.2in}
\noindent {\bf Example 3} Solve the initial-value problem
\[
{\bf x}' = \left(\begin{array}{rr}2&8\\-1&-2\end{array}\right){\bf x},
\hspace{.4in}{\bf x}(0)=
\left(\begin{array}{r}1\\-1\end{array}\right).
\]

\vspace{.2in}
\noindent{\em Solution}. As in the previous example, the system for
$\xi$ is
\[
\left(
\begin{array}{cc}2-r&8\\-1&-2-r\end{array}
\right)\left(\begin{array}{c}\xi_1\\ \xi_2\end{array}\right)=
\left(\begin{array}{c}0\\0\end{array}\right).
\]
The nontrivial solution is obtained by requiring the determinate to
be zero. That is,
\[
(2-r)(-2-r)-(-1)(8)= r^2+4=0.
\]
The eigenvalues (roots) are $r=\pm 2i$. The procedure is the same as
in Example 2. Suppose $r=2i$. The equations for $\xi$ are
\[
\begin{array}{c}(2-2i)\xi_1+8\xi_2=0\\-1\xi_1+(-2-2i)\xi_2=0\end{array}.
\]
It is not obvious the two equations are multiples of one another.
However, you can verify that $(-2+2i)$ times the second equation gives
the first equation. A non-trivial solution is required. If we
opportunistically set $\xi_2=1$ in the second equation, then $\xi_1=
-2-2i$. So
\[
{\bf \xi} = \left(\begin{array}{c}2+2i\\-1\end{array}\right)
\]
will work, and one solution is
\[
{\bf x^{(1)}} = \left(\begin{array}{c}2+2i\\-1\end{array}\right)e^{2i\,t}.
\]

Since the other root (eigenvalue) is just the complex conjugate of the
first, a second solution is found simply by changing $i$ in the first
solution to $-i$ (its complex conjugate). So the general solutions is
\begin{equation}
{\bf x} =
C_1\left(\begin{array}{c}2+2i\\-1\end{array}\right)e^{2i\,t}
+C_2\left(\begin{array}{c}2-2i\\-1\end{array}\right)e^{-2i\,t}.
\label{complex}
\end{equation}
The complex numbers are undesirable. We hide them by using Euler's
formula: $e^{i\theta}=\cos\theta+i\sin\theta$. The exponential
functions in \R{complex} are replaced, and the result expressed
as the real plus imaginary part -
\[
{\bf x} =
C_1\left(\begin{array}{c}2+2i\\-1\end{array}\right)(\cos 2t+i\sin 2t)
+C_2\left(\begin{array}{c}2-2i\\-1\end{array}\right)(\cos 2t-i\sin 2t).
\]
Read across the first line and pick out all the terms without an
$i$. Then do the same for the bottom line (that is, find the real
part). The result is
\[
\mbox{Re}({\bf x}) = (C_1+C_2)\left(\begin{array}{c}
2\cos 2t-2\sin 2t\\
-\cos 2t\end{array}\right).
\]
The imaginary part consists of all the terms with an $i$ in
front. In particular,
\[
i\mbox{Im}({\bf x}) = i(C_1-C_2)\left(\begin{array}{c}
2\cos 2t+2\sin 2t\\
-\sin 2t\end{array}\right).
\]
Finally, set $c_1=C_1+C_2$ and $c_2=i(C_1-C_2)$. Then
\[
{\bf x} = c_1\left(\begin{array}{c}
2\cos 2t-2\sin 2t\\
-\cos 2t\end{array}\right)+
c_2\left(\begin{array}{c}
2\cos 2t+2\sin 2t\\
-\sin 2t\end{array}\right)
\]
is the general solution.

\vspace{.2in}
There is a slight short-cut to this procedure. Since there is so much
symmetry in the solutions for the two roots (they differ by replacing
$i$ with -$i$), one might expect that all the necessary information is
contained in one of the solutions. This is the case. The solution for
$r=2i$ (first part of Equation \R{complex}) is
\[
{\bf x^{(1)}} =
\left(\begin{array}{c}2+2i\\-1\end{array}\right)e^{2i\,t}
=\left(\begin{array}{c}2+2i\\-1\end{array}\right)(\cos 2t+i\sin 2t).
\]
To find the general solution, only the real and imaginary parts of
this solution need to be found. Again rewriting as real plus
imaginary,
\[
{\bf x^{(1)}} =
\left(\begin{array}{c}
2\cos 2t-2\sin 2t\\
-\cos 2t\end{array}\right)+i
\left(\begin{array}{c}
2\cos 2t+2\sin 2t\\
-\sin 2t\end{array}\right).
\]
The general solution is ${\bf x^{(1)}}=c_1\mbox{Re}({\bf x^{(1)}})
+c_2\mbox{Im}({\bf x^{(1)}})$ or
\[
{\bf x} = c_1\left(\begin{array}{c}
2\cos 2t-2\sin 2t\\
-\cos 2t\end{array}\right)+
c_2\left(\begin{array}{c}
2\cos 2t+2\sin 2t\\
-\sin 2t\end{array}\right)
\]
as before.

\vspace{.5in}
\begin{center}{\bf\sc Repeated Eigenvalues}\end{center}
Of course the eigenvalues need not all be distinct. Consider for
example
\begin{equation}
{\bf x}' = \left(\begin{array}{rr}1&-1\\1&3\end{array}\right){\bf x}.
\label{mroot}
\end{equation}
Proceeding as before, the system for ${\bf\xi}$ is
\begin{equation}
\left(
\begin{array}{cc}1-r&-1\\1&3-r\end{array}
\right)\left(\begin{array}{c}\xi_1\\ \xi_2\end{array}\right)=
\left(\begin{array}{c}0\\0\end{array}\right).
\label{xisys}
\end{equation}
The nontrivial solution is obtained by requiring the determinate to
be zero. That is,
\[
(1-r)(3-r)-(-1)= r^2-4r+4=0.
\]
The eigenvalues (roots) are $r=2,2$. The equations for ${\bf\xi}$ are
\[
\begin{array}{c}-\xi_1-\xi_2=0\\\xi_1+\xi_2=0\end{array}.
\]
A nontrivial solution is
\[
{\bf \xi} = \left(\begin{array}{r}1\\-1\end{array}\right),
\]
and one solution is
\[
{\bf x^{(1)}} = \left(\begin{array}{r}1\\-1\end{array}\right)e^{2t}.
\]
In analogy with second-order, linear, constant coefficient,
homogeneous ODEs, we might expect the second solution to be
\begin{equation}
{\bf x^{(2)}} = t{\bf x^{(1)}}
=t\left(\begin{array}{r}1\\-1\end{array}\right)e^{2t}.
\label{hope}
\end{equation}
Unfortunately, this is NOT a solution. What to do?

Somehow a candidate for the second solution has to be constructed from
the only solution we have. Based on previous experiences, \R{hope} has
to be close to the correct solution. We alter it slightly and try
again. Set
\begin{equation}
{\bf x^{(2)}} = t{\bf x^{(1)}}+{\bf\eta}e^{2t}
=t\left(\begin{array}{c}1\\-1\end{array}\right)e^{2t}+
\left(\begin{array}{c}\eta_1\\\eta_2\end{array}\right)e^{2t}.
\label{hope2}
\end{equation}
We have to see if a choice for ${\bf\eta}$ exists so that
${\bf x^{(2)}}$ solves \R{mroot}. The left side of \R{mroot} (the time
derivative of \R{hope2}) is
\begin{equation}
{\bf x^{(2)}}' = {\bf x^{(1)}}+t{\bf
  x^{(1)}}'+\left(\begin{array}{c}\eta_1\\\eta_2\end{array}\right)2e^{2t}
=\left(\begin{array}{c}1\\-1\end{array}\right)e^{2t}+
2t\left(\begin{array}{c}1\\-1\end{array}\right)e^{2t}
+\left(\begin{array}{c}\eta_1\\\eta_2\end{array}\right)2e^{2t}.
\label{ls}
\end{equation}
The right side is
\begin{equation}
\left(\begin{array}{rr}1&-1\\1&3\end{array}\right)
\left(\begin{array}{c}1\\-1\end{array}\right)te^{2t}+
\left(\begin{array}{rr}1&-1\\1&3\end{array}\right)
\left(\begin{array}{c}\eta_1\\\eta_2\end{array}\right)e^{2t}.
\label{rs}
\end{equation}
However,
\[
\left(\begin{array}{rr}1&-1\\1&3\end{array}\right)
\left(\begin{array}{c}1\\-1\end{array}\right)te^{2t}
=2t\left(\begin{array}{c}1\\-1\end{array}\right)e^{2t}.
\]
Equating \R{ls} and \R{rs}, canceling the common $e^{2t}$ and the
other common terms, we find
\[
\left(\begin{array}{c}1\\-1\end{array}\right)+
\left(\begin{array}{c}\eta_1\\\eta_2\end{array}\right)2
=
\left(\begin{array}{rr}1&-1\\1&3\end{array}\right)
\left(\begin{array}{c}\eta_1\\\eta_2\end{array}\right).
\]
After rearranging ($r=2$ here)
\[
\left(\begin{array}{cc}1-r&-1\\1&3-r\end{array}\right)
\left(\begin{array}{c}\eta_1\\\eta_2\end{array}\right)
=\left(\begin{array}{c}1\\-1\end{array}\right)
\]
or more generally the system for ${\bf\eta}$ is
\begin{equation}
\left(\begin{array}{cc}1-r&-1\\1&3-r\end{array}\right)
\left(\begin{array}{c}\eta_1\\\eta_2\end{array}\right)
=\left(\begin{array}{c}\xi_1\\\xi_2\end{array}\right).
\label{etasys}
\end{equation}

Notice \R{etasys} is an iteration of \R{xisys}. For the current
example, \R{etasys} in component form is
\begin{eqnarray*}-\eta_1-\eta_2&=&1\\
\eta_1+\eta_2&=&\!\!\!-1.
\end{eqnarray*}
As for the system for ${\bf\xi}$, the two equations for $\eta$ are
multiples of one another. Any solution will suffice. The simplest
way to find a solution is to set either $\eta_1$ or $\eta_2$ to
zero. Hence, $\eta_1=-1$ and $\eta_2=0$ will work, and the second
solution is
\[
{\bf x^{(2)}}=t\left(\begin{array}{r}1\\-1\end{array}\right)e^{2t}+
\left(\begin{array}{r}-1\\0\end{array}\right)e^{2t},
\]
and the general solution is
\[
{\bf x}=c_1\left(\begin{array}{r}1\\-1\end{array}\right)e^{2t}+
c_2\left[\left(\begin{array}{r}1\\-1\end{array}\right)te^{2t}+
\left(\begin{array}{r}-1\\0\end{array}\right)e^{2t}\right].
\]

\vspace{.2in}
\noindent {\bf Example 4.} Find the solution of the following
system of differential equations.
\[
{\bf x}' = \left(\begin{array}{rr}3&9\\-1&-3\end{array}\right){\bf x},
\hspace{.4in}{\bf x}(0)=
\left(\begin{array}{c}2\\4\end{array}\right).
\]

\vspace{.15in}
\noindent{\em Solution}. The system for $\xi$ is
\[
\left(
\begin{array}{cc}3-r&9\\-1&-3-r\end{array}
\right)\left(\begin{array}{c}\xi_1\\ \xi_2\end{array}\right)=
\left(\begin{array}{c}0\\0\end{array}\right).
\]
The nontrivial solution is obtained by requiring the determinate to
be zero. That is,
\[
(3-r)(-3-r)-(-1)(9)= r^2=0.
\]
The eigenvalues (roots) are $r=0,0$. The equations for ${\bf\xi}$ are
\[
\begin{array}{c}3\xi_1+9\xi_2=0\\1\xi_1+3\xi_2=0\end{array}.
\]
A nontrivial solution is $\xi_1=3$, $\xi_2=-1$, and
\[
{\bf x^{(1)}}=\left(\begin{array}{r}3\\-1\end{array}\right).
\]
The system for ${\bf\eta}$ is
\[
\left(
\begin{array}{cc}3&9\\-1&-3\end{array}
\right)\left(\begin{array}{c}\eta_1\\ \eta_2\end{array}\right)=
\left(\begin{array}{r}3\\-1\end{array}\right).
\]
Any non-trivial solution be sufficient. The choice $\eta_2=0$ and
$\eta_1=1$ is a solution. Therefore, the general solution is
\[
{\bf x}=c_1\left(\begin{array}{r}3\\-1\end{array}\right)+
c_2\left[t\left(\begin{array}{r}3\\-1\end{array}\right)+
\left(\begin{array}{r}1\\0\end{array}\right)\right].
\]

To find the solution, the initial data must be applied. At $t=0$
\[
\left(\begin{array}{c}2\\4\end{array}\right)=
{\bf x}(0)=c_1\left(\begin{array}{r}3\\-1\end{array}\right)+
c_2\left(\begin{array}{r}1\\0\end{array}\right).
\]
We find $c_1=-4$ and $c_2=14$, and the solution is
\[
{\bf x}(t)=\left(\begin{array}{c}2+42t\\-4-14t\end{array}\right).
\]
\subsection{Equilibrium Points}
Consider the autonomous system of ordinary differential equations of the form
\begin{equation}\label{auton}
\begin{aligned}
\frac{dx}{dt}=f(x, y)\\
\frac{dy}{dt}=g(x, y)
\end{aligned}
\end{equation}
\begin{defn}[Equilibrium point]
An equilibrium(a critical) point of the system (\ref{auton}) is a point $(x_*,y_*)$ such that
\[
f(x_*,y_*) = g(x_*,y_*) = 0
\]
\end{defn}
Let $\textbf{x} = (x(t),y(t)),\textbf{x}_0 = (x_0,y_0)$, and $\textbf{x}_* = (x_*,y_*)$.
A critical point $x_*$ is \textbf{stable} provided that, for each $\varepsilon> 0$, there exists $\delta>0$ such that
\[
\|\textbf{x}_0-\textbf{x}_*\| < \delta\Rightarrow \|\textbf{x}(t)-\textbf{x}_*\| <\varepsilon \mbox{ for all } t > 0.
\]

The critical point $(x_*,y_*)$ is called \textbf{unstable} if it is not stable.

A critical point $x_*$ is  \textbf{asymptotically stable} if there exists $\delta>0$ such that
\[
\|\textbf{x}_0-\textbf{x}_*\| < \delta\Rightarrow \lim_{t\to\infty}\textbf{x}(t)=\textbf{x}_*
\]
Let $\lambda_1$ and $\lambda_2$ be the eigenvalues of
\[
J(\textbf{x}_*)=
\begin{pmatrix}
  f_x & f_y \\
  g_x & g_y
\end{pmatrix}\biggl|_{\textbf{x}_*}
\]
Then the stability of the equilibrium points of the system (\ref{auton}) is summarized in table \ref{tab:a}.

\subsection{Stability}
\begin{table}[!htb]
\centering
\caption{Stability properties of linear systems}\label{tab:a}
\begin{tabular}{@{}  l l l @{}}\Xhline{4\arrayrulewidth}
Eigenvalue    & Type of critical point  &  Stability        \\ \Xhline{4\arrayrulewidth}
$\lambda_1,\lambda_2>0$ & Improper node           & Unstable \\
$\lambda_1,\lambda_2<0$ & Improper node           & Asymptotically stable  \\
$\lambda_2<0<\lambda_1>0$ & Saddle point            & Unstable   \\
$\lambda_1=\lambda_2>0$ & Proper or improper node & Unstable  \\
$\lambda_1=\lambda_2<0$ & Proper or improper node & Asymptotically stable \\
$\lambda_1,\lambda_2=\lambda\pm i\mu$ & Spiral point            &   \\
$\lambda>0$  &                         & Unstable \\
$\lambda<0$  &                         & Asymptotically stable  \\
$\lambda_1,\lambda_2=\pm i\mu$ &  Centre                 & Stable  \\
\Xhline{4\arrayrulewidth}
\end{tabular}
\end{table}

Table \ref{tab:a} provides information on the type of stability that could result based on the nature of the eigenvalues of a linear system.

\subsection{The Routh-Hurwitz Criteria}
Given a polynomial
\[
P(\lambda)=\lambda^n+a_1\lambda^{n-1}+a_2\lambda^{n-2}+\cdots+a_n
\]
where the coefficients $a_i$ are real constants,$i=1,2,\ldots,n$ define the $n$ Hurwitz matrices using the coefficients of the characteristic polynomial:
$$
H_1=
\begin{pmatrix}
a_1
\end{pmatrix},
H_2=
\begin{pmatrix}
a_1  & 1  \\
a_3  & a_2
\end{pmatrix},
H_3=
\begin{pmatrix}
a_1  & 1    & 0   \\
a_3  & a_2  & a_1  \\
a_5  & a_4  & a_3
\end{pmatrix},\cdots
$$
and
$$
H_n=
\begin{pmatrix}
a_1  & 1    & 0    & 0    & 0    & 0    & 0    & \cdot & \cdot & 0  \\
a_3  & a_2  & a_1  & 1    & 0    & 0    & 0    & \cdot & \cdot & 0  \\
a_5  & a_4  & a_3  & a_2  & a_1  & 1    & 0    & \cdot & \cdot & 0   \\
a_7  & a_6  & a_5  & a_4  & a_3  & a_2  & a_1  & \cdot & \cdot & \cdot \\
\cdot & \cdot & \cdot & \cdot & \cdot & \cdot & \cdot & \cdot & \cdot & \cdot\\
\cdot & \cdot & \cdot & \cdot & \cdot & \cdot & \cdot & \cdot & \cdot & \cdot \\
0    & 0    & 0    &  0   &  0   & 0    & 0    & \cdot & \cdot & a_n
\end{pmatrix}
$$

where $a_j=0$ for $j>n$. All the roots of polynomial $P(\lambda)$ are negative or have negative real parts iff the determinants of all Hurwitz matrices are positive:
\[\det H_j>0,\quad j=1,2,\ldots,n\]
For $n=2$, $\det H_1=a_1>0$ and $H_2=
\begin{pmatrix}
a_1  & 1  \\
0  & a_2
\end{pmatrix}=a_1a_2>0$
or $a_1>0$ and $a_2>0$.
For polynomials of degree $2,3,4$ and $5$, the Routh Hurwitz Criteria are summarized as follows:
\begin{align}
n=2: &a_1>0 \mbox{ and } a_2>0\\
n=3: &a_1>0,a_3>0 \mbox{ and } a_1a_2-a_3>0\\
n=4: &a_1>0,a_3>0,a_4>0 \mbox{ and } a_1a_2a_3-a_3^2-a_1^2a_4>0\\
n=5: &a_i>0,i=1,2,3,4 \mbox{ and } 5,  a_1a_2a_3-a_3^2-a_1^2a_4>0,
\end{align}
and
\begin{align}
&(a_1a_4-a_5)(a_1a_2a_3-a_3^2-a_1^2a_4)-a_5(a_1a_2-a_3)^2-a_1a_5^2>0
\end{align}
% ------------------------------------------------------------------------------
% End document
% ------------------------------------------------------------------------------
\newpage
\begin{thebibliography}{9}

\bibitem{Jun}
[W.G. Kelley and A.C. Peterson] ~
The Theory of Differential Equations: Classical and Qualitative.

\bibitem{July}
[Coddington and Levinson] ~
Theory of Ordinary Differential Equations, McGraw-Hill, New York, 1955.

\end{thebibliography}
\end{document}
